Durante el transcurso del desarrollo de oFlute, y sobre todo al término del
mismo, se han obtenido unas conclusiones y unos resultados, tanto de forma
personal como para con la comunidad, que intentaremos reflejar en este capítulo.

oFlute ha sido el proyecto más longevo al que me he enfrentado hasta ahora. A
pesar de tener conciencia de la envergadura del mismo desde el principio, el
tiempo para completarlo ha superado todas mis expectativas, sobre todo en lo que
a documentación se refiere. La causa de esto ha sido, por un lado, una
planificación inicial algo ineficiente y, por otro lado, mi recelo sobre los
procedimientos actuales sobre ingeniería del software. De cualquier modo, estoy
bastante satisfecho con lo obtenido, y esto me ha servido para aprender a
regirme por una planificación de forma más estricta.

Gracias a oFlute he aprendido las técnicas básicas de la programación de audio,
sobre todo en la parte técnica más que teórica. Es esta parte teórica, sobre
todo la de análisis, la que me gustaría reforzar en un futuro, ahora que cuento
con las bases para conseguirlo. Como se comentó en el capítulo sobre
\textit{Investigación preliminar}, los videojuegos relacionados con el audio son
un nicho aún poco explorado y que puede dar muchas satisfacciones, sobre todo
cuando el producto se orienta a un público joven.

Por otro lado, oFlute también me ha ayudado a aprender a usar algunas
tecnologías secundarias, en algunos casos incluso obteniendo un conocimiento
suficiente para generar documentación e impartir talleres relacionados con ello.

Una de estas tecnologías es \textbf{Boost}~\cite{boost}, un conjunto de
bibliotecas para C++ que amplían en gran medida la biblioteca estándar del
lenguaje. Un amplio número de componentes de Boost formarán parte del nuevo
estándar C++0x~\cite{cpp0x}, por lo que me ha servido para ponerme al día en las
novedades que están por llegar.

Utilizando como material de apoyo el libro \textit{``Beyond the C++ Standard
  Library''}~\cite{libroboost}, pude conocer una gran parte de las bibliotecas
  de Boost, incluyendo punteros inteligentes, programación funcional,
  bibliotecas para hilos, y utilizarlas en el proyecto. Además, también apliqué
  este nuevo conocimiento en el desarrollo de un taller durante los Cursos de
  Verano de la OSLUCA~\cite{cursosverano}, en el que se explicaron las partes más
  importantes de Boost, con numerosos ejemplos de cada una. Toda la
  documentación es libre~\cite{materialesCursoBoost}.

Al tratarse de la principal biblioteca utilizada durante el desarrollo, oFlute
me ha provisto de un profundo conocimiento de \textbf{Gosu}~\cite{gosu},
permitiéndome implementar videojuegos con mucha más fluidez y labrándome un
pequeño \textit{framework} personal que utilizar de base en próximos
proyectos. A raíz de esto impartí un taller~\cite{tallergosu} sobre la
biblioteca en colaboración con la ADVUCA~\cite{advuca}, cuya
afluencia superó las 50 personas. Los materiales pueden descargarse
libremente~\cite{tallergosumateriales}.

Uno de estos proyectos \textit{``hijos''} de oFlute ha sido
\textbf{Freegemas}~\cite{freegemas}, un clon libre del popular juego tipo puzzle
\textit{Bejeweled}. Freegemas es multiplataforma, funciona tanto en GNU/Linux
como en Windows, y además forma parte oficial de Guadalinex~\cite{guadalinex},
por lo que es posible encontrarlo en los repositorios oficiales. 

\begin{figure}[h!]
  \centering
  \includegraphics[width=0.55\textwidth]{8_conclusiones/imagen_freegemas}
  \caption{Logotipo de Freegemas}
\end{figure}

Además, Freegemas sirvió como base para una serie de tres artículos que
publiqué, junto al doctor Manuel Palomo Duarte, en la revista Linux
Magazine~\cite{linuxmagazine} sobre desarrollo de videojuegos en C++. Es posible
encontrar estos artículos en el archivo de la
revista~\cite{refarticulo1}\cite{refarticulo2}\cite{refarticulo3} bajo una
licencia Creative Commons.

Otra de las tecnologías que he aprendido ha sido \textbf{GNU Gettext}, que ha
servido para internacionalizar el proyecto. A este efecto escribí una guía
concisa sobre traducción de proyectos con esta herramienta, que se puede
encontrar en el apéndice~\textit{\nameref{sec:gettext}}.

Por otro lado, gracias a oFlute tuve la oportunidad de participar en el IV
Concurso Universitario de Software Libre~\cite{cusl}. En el transcurso del
concurso formé parte de una comunidad muy unida, en la que reinó el apoyo y la
ayuda entre los concursantes. La final del concurso se celebró en la Escuela
Superior de Ingeniería de Cádiz, en la que oFlute obtuvo una mención
especial~\cite{cusl2}.

\begin{figure}[h!]
  \centering
  \includegraphics[width=0.65\textwidth]{8_conclusiones/imagen_logocusl}
  \caption{IV Concurso Universitario de Software Libre}
\end{figure}

Previa a la final nacional tuvo lugar la fase local del concurso, en el que el
proyecto también recibió un accésit al mejor proyecto de
innovación~\cite{cusllocal}.
