\documentclass[landscape]{article}

\usepackage[utf8]{inputenc}
\usepackage[spanish]{babel}

\usepackage[top=1cm, bottom=1cm, left=1cm, right=1cm]{geometry}

\usepackage{multicol}
\setlength{\columnsep}{20pt}

\renewcommand{\familydefault}{\sfdefault}

\usepackage{setspace}
\onehalfspacing

\setlength{\parskip}{0.4cm}

\title{Apuntes para la presentación}
\author{José Tomás Tocino García}
\date{}

\usepackage{framed}

\newenvironment{nota}
{% This is the begin code
\begin{framed} \noindent\itshape
}
{% This is the end code
\end{framed}\vspace{-0.5cm} }

\begin{document}

\pagestyle{empty}
\begin{multicols*}{2}

\begin{center}
  \begin{Large}\textbf{Apuntes para la presentación del PFC}\end{Large}\\
  \begin{large}José Tomás Tocino García\end{large}\\[0.1cm]
\end{center}

Buenos días, mi nombre es José Tomás Tocino García, soy alumno de Ingeniería
Técnica en Informática de Sistemas, y voy a presentar mi proyecto titulado
``\textit{oFlute: reconocimiento de señales aplicado al aprendizaje de la flauta
  dulce}''.

\begin{nota}
  Pasar a transparencia 2: índice.
\end{nota}

Este es el índice que vamos a seguir durante la presentación. \textbf{EXPLICAR
  ÍNDICE}

\begin{nota}
  Pasar a transparencia 3: contexto social. \\Aparece ``Jóvenes en plena simbiosis con
  las nuevas tecnologías.''.
\end{nota}

En la actualidad, las nuevas generaciones están en plena simbiosis con las
tecnologías de la información. Ya sea mediante redes sociales, videojuegos o
cualquier otra clase de sistema multimedia, desde muy jóvenes se acostumbran al
empleo de dispositivos electrónicos, haciendo que su uso sea prácticamente
instintivo.

\begin{nota}
  Step. Aparece `` Las TIC están llegando a los centros educativos.''
\end{nota}

Por otro lado, las nuevas tecnologías van filtrándose gradualmente en los
centros educativos. Muestra de elo es el reparto de ordenadores portátiles a los
alumnos andaluces de 5o y 6o de primaria, dentro del marco de la Escuela TIC
2.0, que ha llevado a cabo la Junta de Andalucía.

\begin{nota}
  Step. Aparece ``Técnicas docentes basadas en recursos multimedia e informáticos.''
\end{nota}

Estos dos factores favorecen la aparición de nuevas técnicas docentes, basadas
en el uso de recursos multimedia y equipos informáticos. Estas técnicas tienen
una doble ventaja. Por un lado, facilitan a los profesores el impartir su
temario, y por otro lado, resultan atractivas para los alumnos, que a menudo
muestran más interés y adquieren más fácilmente el conocimiento en comparación
con las técnicas tradicionales.

\begin{nota}
  Pasar. Transparencia ``Concepción del proyecto''.
\end{nota}

Por ello, se tomó la decisión de desarrollar un videojuego educativo, que
pudiera utilizarse fácilmente en las escuelas y que tuviera una utilidad real
con respecto al aprendizaje de los alumnos. Actualmente se trata de un sector
innovador, en pleno crecimiento y con vistas a un futuro prometedor.

\begin{nota}
  Step.
\end{nota}

La primera cuestión que surgió fue el tema del proyecto. ¿Sobre qué aspecto
educativo debería versar la aplicación? ¿Qué asignatura se beneficiaría del
desarrollo?

Estuvimos pensando en las diferentes materias que se imparten en los niveles de
primaria, viendo cuál de ellas podría beneficiarse más de un proyecto de esta
clase, y finalmente se decidió orientar la aplicación hacia la música.

\begin{nota}
  Step.
\end{nota}

Dentro de la música había bastantes aspectos que podrían servir, pero el que nos
resultó más atractivo fue el \framebox{STEP} aprendizaje de la flauta dulce. Es
un instrumento económico y fácil de aprender que siempre se ha utilizado en las
clases de música.

\vfill
\pagebreak


\end{multicols*}
\end{document}