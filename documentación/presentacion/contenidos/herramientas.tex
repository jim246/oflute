\begin{frame}{Herramientas}
  \begin{block}{Lenguaje de programación: C++}
    \begin{description}
    \item[Pros]     
      \begin{itemize}
      \item Mayor familiaridad.
      \item Muy eficiente.
      \item Gran cantidad de herramientas y soporte.
      \end{itemize}
      \medskip
    \item[Contras]
      \begin{itemize}
      \item Desarrollo más lento que en lenguajes de script.
      \item Gestión de memoria manual.
      \end{itemize}
    \end{description}
  \end{block}
\end{frame}

\begin{frame}{Herramientas}
  \begin{block}{Biblioteca gráfica: Gosu}
    \begin{description}
    \item[Pros]
      \begin{itemize}
      \item Multiplataforma.
      \item Muy orientada a objetos.
      \item Aceleración gráfica por hardware.
      \end{itemize}

      \medskip


    \item[Contras]
      \begin{itemize}
      \item Alcance limitado: solo gráficos y E/S.
      \item Inconsistencias entre sistemas.
      \item Poco soporte.
      \end{itemize}

    \end{description}
  \end{block}
\end{frame}

\begin{frame}{Herramientas}
  \begin{block}{Acceso a flujos de audio}
    PulseAudio, muy bajo nivel.
  \end{block}

  \begin{block}{Procesado de XML}
    PugiXML, sencilla y rápida, con acceso XPath.
  \end{block}

  \begin{block}{Cálculo de DFT}
    Tras probar una implementación propia, se pasó a KissFFT, por eficiencia.
  \end{block}

  \begin{block}{Propósito general}
    Boost se utilizó de forma extensa.
  \end{block}
\end{frame}

%%% Local Variables: 
%%% mode: latex
%%% TeX-master: "../presentacion"
%%% End: 
