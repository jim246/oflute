\documentclass[a4paper, 12pt, halfparskip]{scrbook}

\usepackage{graphicx}
\usepackage[spanish]{babel}
\usepackage[utf8]{inputenc}
\usepackage{hyphenat}

\usepackage{charter}

\usepackage[scaled=0.92]{helvet}

\begin{document}
\chapter{Introducción}
\section{Contexto y motivación}
Las nuevas tecnologías van filtrándose gradualmente en los centros
educativos, y las técnicas de enseñanza se están adaptando a las
opciones que ofrecen. El reparto de ordenadores portátiles a los
alumnos andaluces de 5º y 6º de primaria, dentro del marco de la
Escuela TIC 2.0, es buena muestra de ello. 

Por otro lado, las nuevas generaciones están en plena simbiosis con
las tecnologías de la información, cada vez más acostumbradas al
empleo de dispositivos electrónicos, y su uso ya les es prácticamente
instintivo. Por tanto, es beneficioso buscar nuevos métodos educativos
que hagan uso de las nuevas tecnologías.

En la búsqueda de materias educativas en las que aplicar el uso de las
nuevas tecnologías, la música, parte fundamental del programa
curricular en la educación primaria, ofrece una gran variedad de
aspectos que podrían desarrollarse utilizando tecnologías de la
información. Es ahí donde este proyecto hace su aportación.

\section{Objetivos}
A la hora de definir los objetivos de un sistema, podemos agruparlos
en dos tipos diferentes: \textbf{funcionales} y
\textbf{transversales}. Los primeros se refieren a \textit{qué} debe
hacer la aplicación que vamos a desarrollar, e inciden
directamente en la experiencia del usuario y de potenciales
desarrolladores.

Por otro lado, los objetivos transversales son aquellos invisibles al
usuario final, pero que de forma inherente actúan sobre el resultado
final de la aplicación y sobre la experiencia de desarrollo de la misma.

\subsection{Funcionales}
\begin{itemize}
\item Crear un módulo de análisis del sonido en el dominio de la
  frecuencia para poder identificar las notas capturadas por el
  micrófono en tiempo real.
\item Crear una aplicación de usuario que identifique y muestre en
  pantalla las notas que toca el usuario en cada momento.
\item Reutilizar el módulo de análisis en un juego en el que el
  usuario debe tocar correctamente las nota que aparecen en pantalla
  siguiendo un pentagrama.
\item Incluir un sistema de lecciones multimedia individuales que
  sirvan al alumno de referencia y fuente de aprendizaje.
\item Potenciar el uso de interfaces de usuario amigables, con un
  sistema avanzado de animaciones que proporcione un aspecto fluido y
  evite saltos bruscos entre secciones.
\end{itemize}

\subsection{Transversales}
\begin{itemize}
\item Obtener una base teórica sobre cómo se representa y caracteriza
  digitalmente el sonido.
\item Conocer las bases del \textbf{DSP}, y su uso en aplicaciones de
  reconocimiento básico de sonidos, tales como sintonizadores y
  afinadores de instrumentos.
\item Introducirme en la programación de audio en sistemas GNU/Linux.
\item Entender las bases del análisis de sonidos en el dominio de la
  frecuencia. 
\item Utilizar un enfoque de análisis, diseño y codificación orientado
  a objetos, de una forma lo más clara y modular posible, para
  permitir ampliaciones y modificaciones sobre la aplicación por
  terceras personas.
\item Hacer uso de herramientas básicas en el desarrollo de software,
  como son los \textbf{Sistemas de Control de Versiones} para llevar
  un control realista del desarrollo del software, así como hacer de
  las veces de sistema de copias de seguridad.
\end{itemize}



\end{document}
