\documentclass[a4paper]{scrartcl}

\usepackage[utf8]{inputenc}
\usepackage[spanish]{babel}

\usepackage{hyperref}
\usepackage{graphicx}

\usepackage[spanish]{varioref}
\usepackage{wrapfig}

\usepackage[top=1.5cm, bottom=2cm]{geometry}

\title{Introducción a OGRE3D}
\author{José Tomás Tocino García}
\date{\today}

\begin{document}

\maketitle
\section*{¿Qué es OGRE3D?}

\begin{wrapfigure}{r}{3cm}
\vspace{-1cm}
  \includegraphics[scale=0.3]{caraOgro}
\vspace{-0.5cm}
\end{wrapfigure}

OGRE (\textit{Object-Oriented Graphics Rendering Engine}) es un motor de renderizado 3D orientado a objetos y escrito en C++.
Es multiplataforma y se puede utilizar tanto con \textbf{Direct3D} como con \textbf{OpenGL}. Aunque sea mayoritariamente utilizado 
a la hora de desarrollar juegos, hay que tener en cuenta que no se trata de un motor de videojuegos, como podría serlo la \textbf{SDL},
sino que sólo incluye el motor de renderizado y es necesario utilizar terceras librerías para el manejo del audio, el control, etc.

\section*{Antes de empezar...}
Aquellos que no conozcan la programación orientada a objetos y/o no controlen del todo el lenguaje C++, deberían dirigirse a~\cite{DBLP:books/aw/Stroustrup91}.
Por otro lado, dado que en OGRE se utilizan un gran número de patrones de diseño, sería beneficiosa la lectura de la referencia \cite{GammaHJV93}.

\section*{Clases principales}
OGRE tiene una fuerte orientación a objetos y es ese aspecto el que a menudo dificulta su aprendizaje. Las principales clases
del motor son las siguientes:

\begin{table}[h!]
  \centering
  \begin{tabular}{|l||p{10cm}|}
  \hline
	\textit{Clase} & \textit{Descripción}\\
	\hline
	\texttt{Ogre::Root} & Es el punto de inicio para cualquier aplicación de OGRE.\\
	\hline
	\texttt{Ogre::RenderWindow} & Se encarga de la ventana principal en la que se producirá el renderizado.\\
	\hline
	\texttt{Ogre::SceneManager} & Se encarga de la organización de la escena, así como de su renderizado.\\
	\hline
	\texttt{Ogre::Camera} & Representa una cámara.\\
	\hline
	\texttt{Ogre::Viewport} & Un Viewport es una zona de la ventana en la que se realiza un renderizado.\\
  \hline
	\texttt{Ogre::ResourceGroupManager} & Singleton que se encarga de organizar los grupos de recursos, su carga y descarga, etc.\\
	\hline
	\texttt{Ogre::Entity} & La clase más importante: Representa los modelos 3D interactivos (que se pueden mover).\\
	\hline
  \end{tabular}
  \caption{Principales clases de OGRE}
  \label{tab:tablaclases}
\end{table}

Ahora haremos un repaso a las clases comentadas en la anterior tabla~\vref{tab:tablaclases}. \textbf{To be continued...}

\bibliographystyle{alpha}
\bibliography{bibliografia}
\end{document}