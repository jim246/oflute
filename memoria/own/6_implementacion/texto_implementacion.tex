Un análisis claro y un diseño conciso no garantizan que, a la hora de
implementar el sistema planteado, no se encuentre ninguna dificultad o
imprevisto. Así pues, en este capítulo comentaremos los retos y detalles que más
relevancia o complejidad han presentado durante la fase de la implementación del
proyecto.

De igual modo, durante el desarrollo de la aplicacíon se mantuvo actualizada una
bitácora, accesible en línea\footnote{\url{http://oflute.wordpress.com}},
en la que se fueron detallando, a medida que aparecían, muchas de estas
dificultades. 

Como complemento a la lectura de este capítulo se recomienda tener una copia
local del repositorio del proyecto, disponible para su libre descarga desde la
forja oficial\footnote{\url{http://oflute.googlecode.com}}. En él se encuentra
todo el código fuente original, así como la documentación en formato
\textit{Doxygen}.

\section{Carga y uso de fuentes TrueType en Gosu}

Como se ha comentado previamente, \textbf{oFlute} hace uso de la biblioteca
\textit{Gosu}, que proporciona una API sencilla para el acceso al sistema
gráfico, entre otras características. Este framework funciona en sistemas
Windows, GNU/Linux y Mac OS, aunque dada la dificultad de conseguir la
compatibilidad con todos ellos, la calidad y el rendimiento es bastante
desigual.

Una de estas desigualdades se presentaba a la hora de cargar fuentes para
mostrar textos. En su versión para GNU/Linux, Gosu \textbf{no permitía} utilizar
fuentes que no estuviesen instaladas en el sistema, esto es, era imposible
adjuntar un fichero con una fuente en formato \texttt{TrueType}
\marginpar{REFERENCIA} para su carga en el juego. Sin embargo, tanto en
Windows como en Mac OS esta carga sí era posible. Esto supuso un grave problema
en el planteamiento del proyecto.

Inicialmente se investigaron las razones de esta limitación. Las conclusiones
que se sacaron fueron que Gosu, bajo GNU/Linux, implementaba el renderizado de
fuentes mediante una biblioteca llamada Pango \marginpar{REFERENCIA}, de
bastante bajo nivel, y que por diseño está limitada al uso de fuentes de
sistema, ya que su uso se orienta a herramientas del sistema operativo, no a
aplicativos de terceros. 

Así pues, era necesario buscar una alternativa. Basándonos en la experiencia
previa con otras bibliotecas de desarrollo, se pensó en \texttt{SDL\_ttf}, una de
las partes de la conocida biblioteca SDL \marginpar{REFERENCIA}, ampliamente
utilizada en el desarrollo de aplicaciones multimedia en toda clases de
sistemas.
