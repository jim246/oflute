\begin{frame}{Conclusiones a nivel de proyecto}
  \begin{block}{Objetivos cumplidos}
    Se completaron todos los objetivos propuestos:
    \begin{itemize}
    \item Se creó un módulo de análisis de notas eficiente.
    \item El sistema de canciones integró el módulo de análisis de forma
      efectiva.
    \item Desarrollamos un sistema de lecciones muy completo.
    \item Se mantuvo en todo momento una interfaz agradable y fluida.
    \end{itemize}    
  \end{block} 
\end{frame}

\begin{frame}{Conclusiones a nivel de proyecto}
  \begin{block}{Posibles mejoras}
    Hay lugar para ampliar el proyecto:
    \begin{itemize}
    \item Extender el sistema de lecciones para añadir, por ejemplo, vídeos y
      otros elementos multimedia.
    \item Mejorar la jugabilidad del sistema de canciones.
    \item Portar el juego a otras plataformas.
    \end{itemize}
  \end{block}  
\end{frame}

\begin{frame}{Conclusiones a nivel personal}
  \begin{itemize}
  \item Proyecto muy longevo.
  \item Mucho conocimiento nuevo adquirido: DSP, programación de audio, hilos,
    matemáticas...
  \item Mucho conocimiento generado.
  \item Cercano a proyectos reales.
  \end{itemize}
\end{frame}

\begin{frame}{Conocimiento generado}
  Se ha generado mucho conocimiento a raíz del proyecto. 

  \pause

  \begin{block}{Taller de Boost}
    Se explicaron las partes más importantes de esta colección de bibliotecas,
    con numerosos ejemplos.
  \end{block}

  \pause

  \begin{block}{Taller de Gosu}
    Afluencia de más de 50 personas, se desarrolló un clon del Arkanoid.
  \end{block}

  \pause

  \begin{block}{Tutorial de Gettext}
    Completo manual de internacionalización de proyectos. También se hizo un
    taller sobre el mismo tema.
  \end{block}
\end{frame}

\begin{frame}{Proyectos derivados}

  \begin{center}
    A partir del código de oFlute se desarrolló el proyecto \textbf{Freegemas}, un clon
    libre y multiplataforma de Bejeweled.

    \medskip

    \includegraphics[width=0.49\textwidth]{imagenes/imagen_freegemas1}\hspace{0.1cm}
    \includegraphics[width=0.49\textwidth]{imagenes/imagen_freegemas2}

    \medskip

    Su desarrollo dio lugar a \textbf{tres publicaciones} en la revista Linux
    Magazine, y su inclusión oficial en \textbf{Guadalinex.}
  \end{center}

\end{frame}

\begin{frame}{Difusión}
  
\end{frame}
%%% Local Variables: 
%%% mode: latex
%%% TeX-master: "../presentacion"
%%% End: 
